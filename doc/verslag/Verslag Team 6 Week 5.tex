\documentclass{article}


\usepackage[urlcolor=blue, linkcolor=black, colorlinks=true]{hyperref}
\usepackage[dutch]{babel}
\usepackage{minutes}

\renewcommand{\familydefault}{\sfdefault}


\makeatletter
\addto\extrasdutch{%
\def\min@textTask{Taak}%
}
\makeatother%
\makeatletter
\def\blfootnote{\gdef\@thefnmark{}\@footnotetext}
\makeatother




\newcounter{team}

% TODO: Enter team number here:
\setcounter{team}{6}


\begin{document}
%	\tableofcontents

	\begin{Minutes}{Programming Project Databases \\ Wekelijks Verslag Team \arabic{team}}
		\moderation{Tim Sanders}
		\minutetaker{Arno Deceuninck}
		\participant{Arno Deceuninck, Sam Peeters, Sien Nuyens, Tim Sanders}
		\missingNoExcuse{/}
		\missingExcused{/}
%		\guest{\ldots}
		\minutesdate{26 februari 2020}
		\starttime{11u00}
		\endtime{11u30}

		\maketitle

		\topic{Coordinator Vergadering}
		Niet geweest wegens presentatie

		\topic{Presentatie}
		\begin{itemize}
		    \item Duidelijke planning maken. Een planning zoals trello raden ze wel aan, aangezien we op lange termijn moeten kunnen plannen. Tegen 19 april zou het af moeten zijn.
		    \item Joey gaf nog het idee om de gevonden routes_drive op match te sorteren en playlists voor de route toe te voegen.
		    \item Technologi"en liggen vast, maar kunnen nog uitbereiden.
		\end{itemize}

		\topic{Status}

			\subtopic{Overzicht Taken}
				\task*[DONE]{Voorbereidingen presentatie}
				\task*[DONE]{Tussentijds rapport}




		\topic{Besproken Onderwerpen}
				\subtopic{Wie wordt de coordinator deze week?}
				    Tim is de enige persoon die nog geen co"ordinator is geweest, dus het is vanzelfsprekend dat hij het nu is.
				\subtopic{Avatars}
			        Elke user heeft een avatar gebasseerd op zijn/haar username.
			    \subtopic{Volgende deadline}
			        We hebben nog 6 weken tot 19 april.
			        De features die nog moeten komen zijn:
			        \begin{itemize}
			            \item Route toevoegen
			            \item Route zoeken
			            \item Settings aanpassen
			            \item Request sturen om met route mee te gaan (incl. accept/reject). Hou er rekening mee dat er een maximum is van mensen die in de auto
			            \item Locatie op map weergeven
			        \end{itemize}
			        Tegen week 5 willen we routes_drive kunnen toevoegen, en de mockups al hun gegevens uit de database laten halen.
			        Tegen week 6 kan je routes_drive zoeken en route requests weigeren/bevestiggen.
			        Tegen week 7 moet aangepast worden en de map moet correct weergegeven worden.
			        Week 8 is een reserve week voor als er iets is misgelopen.

		\topic{Afspraken \& Planning}
			\begin{itemize}
			    \item Iedereen zorgt voor zijn mockups dat ze alle info uit de database halen. (en je moet hiervoor dus waarschijnlijk de database uitbreiden)
			    \item Arno bedenkt nog wel iets random dat hij gaat doen
			\end{itemize}


		\topic{Varia}
			\emph{Eventuele varia punten. Hierbij vragen we ook aan alle aanwezigen of zij nog iets te zeggen hebben.}
				\subtopic{Sien}
				    Niets te zeggen.
				\subtopic{Sam}
				    Best ook rekening houden met deadlines. Als er iets niet op tijd afgeraakt hebben we altijd nog reserves weken.
				\subtopic{Tim}
				    Doe allemaal zeker jullie best en maak Tim trots.
				\subtopic{Arno}
				    Nogmaals, als je iets niet begrijpt, bekijk the Flask Mega Tutorial. Voor de rest ben ik vergeten wat ik nog wou zeggen. Ah ik weet het weer: We willen mails kunnen versturen om je wachtwoord opnieuw aan te vragen, het probleem hierbij is dat je je Google logingegevens in de code moet zetten. We maken dus best een apart Google account hiervoor aan. We gaan testen of dit adres nog vrij is: noreply.team6.ppdb@gmail.com.


		\blfootnote{
			\href{%
				mailto:joey.depauw@uantwerpen.be%
				?subject=PPDB 2019-2020: Wekelijks Verslag Team \arabic{team}%
				&body=Liefste Joey\%0D\%0A%
				\%0D\%0A%
				Gelieve ons wekelijks verslag terug te vinden in de bijlage.\%0D\%0A%
				\%0D\%0A%
				Groetjes\%0D\%0A%
				Je favoriete team: Team \arabic{team}\%0D\%0A%
			}{Klik hier} om mij op te sturen.
		}


	\end{Minutes}
\end{document}